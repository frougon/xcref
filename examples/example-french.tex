\documentclass[french]{article}
\usepackage{lmodern}
\usepackage[T1]{fontenc}
\usepackage[utf8]{inputenc}
\usepackage{babel}
\usepackage{amsthm}
\usepackage{hyperref}
\usepackage{cleveref}
\usepackage{xcref}

\newtheorem{problem}{Problème}
\newtheorem{assertion}[problem]{Assertion}
\newtheorem{proposition}[problem]{Proposition}

\crefname{problem}{problem}{problème}
\crefname{assertion}{assertion}{assertions}
\crefname{proposition}{proposition}{propositions}

% Pour montrer que cette directive de formatage est respectée par xcref
\creflabelformat{assertion}{(#2#1#3)}

\setlength{\parindent}{0pt}

% Wrappers pratiques qui évitent l'utilisation d'options assez longues telles
% que french/preposition=d'après (avec ces wrappers, l'argument optionnel est
% interprété par pgfkeys avec le chemin par défaut /xcref/french). Les options
% situées directement dans /xcref ne sont donc pas accessibles via cet
% argument optionnel, d'où la nécessité de créer deux commandes différentes
% pour les variantes avec ou sans capitale initiale (on pourrait également
% s'en sortir avec une commande acceptant un argument de plus).
\let\fxcref=\xcreffrenchwrapper
\let\fXcref=\xcreffrenchWrapper

% Exemple de configuration d'xcref (désactivée) ; ôtez le \iffalse et le \fi
% pour l'activer.
\iffalse

\xcrefset{
  french/.cd,                  % va dans le « répertoire » /xcref/french
  % 'names table' ne peut être utilisée que dans le préambule. Les deux
  % premiers <item>s de chaque valeur (préposition et article) doivent être
  % sous forme LICR (par exemple, « \`a » au lieu de « à »).
  names table = {
    problem    = {\`a}{le}{problème}{problèmes},
    proposition= {\`a}{la}{proposition}{propositions},
    assertion  = {d'apr\`es}{l'}{assertion}{assertions},
    theorem    = {\`a}{le}{théorème}{théorèmes},
    lemma      = {\`a}{le}{lemme}{lemmes},
    definition = {dans}{la}{définition}{définitions},
  },
  composition table for prepositions and articles = {
    % Seules les prépositions se combinant de manière irrégulière avec les
    % diverses formes de l'article défini doivent être listées ici. Par
    % exemple, il n'est pas utile d'inclure une entrée pour la locution
    % « d'apr\`es » car les composés « d'après l' » (non suivi d'espace),
    % « d'après le », « d'après la » et « d'après les » (les trois derniers
    % suivis d'une espace inter-mots) sont parfaitement réguliers.
    %
    % Chaque clé (avant le signe « = ») doit être sous forme LICR. Il n'est
    % pas nécessaire d'utiliser \@tabacckludge dans les valeurs (et surtout
    % pas dans les clés) car les caractères accentués, qui sont gérés
    % par inputenc ici, l'utilisent automatiquement (voir la
    % documentation d'inputenc et xcref-french.tex pour plus d'infos sur
    % \@tabacckludge).
    \`a    = {à l'}{au }{à la }{aux },
  }
}

\fi

% Plutôt que de redéfinir d'un coup toute la '/xcref/french/names table', il
% est possible de procéder par ajouts successifs de cette manière (ceci remplace
% l'entrée pour le type de référence « section » si elle existe déjà) :
\xcreffrenchSetNamesTableEntry{section}{\`a}{la}{section}{sections}
% \xcreffrenchSetNamesTableEntry{theorem}{\`a}{le}{théorème}{théorèmes}

\begin{document}

\section{Texte bidon avec des labels}

\begin{problem}\label{prob-a}
Ici, nous avons indiqué \verb|\label{prob-a}|.
\end{problem}
\begin{problem}\label{prob-b}
Ici, nous avons indiqué \verb|\label{prob-b}|.
\end{problem}
\begin{problem}\label{prob-c}
Ici, nous avons indiqué \verb|\label{prob-c}|.
\end{problem}
\begin{problem}\label{prob-d}
Ici, nous avons indiqué \verb|\label{prob-d}|.
\end{problem}
\begin{assertion}\label{assertion-a}
Ici, nous avons indiqué \verb|\label{assertion-a}|.
\end{assertion}
\begin{assertion}\label{assertion-b}
Ici, nous avons indiqué \verb|\label{assertion-b}|.
\end{assertion}
\begin{assertion}\label{assertion-c}
Ici, nous avons indiqué \verb|\label{assertion-c}|.
\end{assertion}
\begin{proposition}\label{propo}
Ici, nous avons indiqué \verb|\label{propo}|.
\end{proposition}

Ces énoncés vont être référencés via \textsf{xcref} \fxcref{sec-exemples}.

\section{Exemples}
\label{sec-exemples}

Test avec \verb|\cref| et \verb|\Cref|: \cref{assertion-a} (obtenu avec
\verb|\cref{assertion-a}|), \Cref{assertion-a} (obtenu avec
\verb|\Cref{assertion-a}|). Notez que nous avons utilisé :
\begin{verbatim}
\creflabelformat{assertion}{(#2#1#3)}
\end{verbatim}
afin de montrer qu'\textsf{xcref} préserve cette directive de formatage (voir
ci-dessous). Ceci explique les parenthèses autour des numéros dans les
références de type \og assertion \fg.

\medskip
\fxcref{prob-a,prob-b}\\
\xcref[hyperlinks]{prob-a,prob-b}~\texttt{==}~%
{% Other way to to the same:
 \xcrefset{hyperlinks=true}% lasts for the duration of the current group
 \fxcref{prob-a,prob-b}%
}\\
\fXcref{prob-a}\\
\fxcref[form=article+noun]{prob-a,prob-b}\\
\fxcref[form=noun]{prob-a,prob-b}

\medskip
\fxcref[form=noun]{prob-a,prob-b,prob-c,prob-d,assertion-a,assertion-b}\\
\fxcref[form=article+noun]{prob-a,prob-b,prob-c,prob-d,assertion-a,assertion-b}\\
\fxcref{prob-a,prob-b,prob-c,prob-d,assertion-a,assertion-b}\\
\fXcref{prob-a,prob-b,prob-c,prob-d,propo,assertion-a,assertion-b}\\
\fxcref[preposition=à]{assertion-a}, \fxcref[preposition=d'après]{assertion-b},
\fxcref[preposition=dans]{assertion-b}\\
\fxcref[preposition=d'après]{propo} et \fxcref[form=article+noun]{assertion-a}

\end{document}
