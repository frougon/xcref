\documentclass[ngerman]{article}
\usepackage{lmodern}
\usepackage[T1]{fontenc}
\usepackage[utf8]{inputenc}
\usepackage{babel}
\usepackage{amsthm}
\usepackage{parskip}            % just for presentation
\usepackage{xcolor}
\usepackage{hyperref}
\usepackage{cleveref}
\usepackage{xcref}

\definecolor{MyBlue}{HTML}{1A36D6}
\hypersetup{colorlinks=true,linkcolor=MyBlue}

\newtheorem{problem}{Problem}
\newtheorem{satz}[problem]{Satz}
\newtheorem{proposition}[problem]{Proposition}

\crefname{problem}{Problem}{Probleme}
\crefname{satz}{Satz}{Sätze}
\crefname{proposition}{Proposition}{Propositionen}

% We'll show that this is respected by xcref
\creflabelformat{satz}{#2(#1)#3}

% Convenience wrappers for \xcref that avoid the need to write long options
% such as ngerman/preposition=aus or ngerman/case=dativ (with these wrappers,
% the optional argument is processed with a default path of /xcref/french).
% The options directly in /xcref are thus not accessible via the optional
% argument, hence the need for two different commands for the
% capitalized/uncapitalized variants (a command accepting one more argument
% would also do).
\let\gxcref=\xcrefngermanwrapper
\let\gXcref=\xcrefngermanWrapper

% Sample xcref configuration: remove the \iffalse and \fi lines to activate it.
\iffalse
\xcrefset{
  ngerman/.cd,                  % change to “directory” /xcref/ngerman
  % 'names table' can only be set in the preamble. The prepositions and
  % articles in this mapping must be written in LICR representation (for
  % instance, use \"uber but not über nor {\"u}ber). However, in the text,
  % both \xcref[preposition=\"uber]{...} and \xcref[preposition=über]{...}
  % would be fine.
  names table = {
    problem    = {in}{{das}{Problem}{die}{Probleme}}%
                   {{das}{Problem}{die}{Probleme}}%
                   {{dem}{Problem}{den}{Problemen}}%
                   {{des}{Problems}{der}{Probleme}},
    proposition= {nach}{{die}{Proposition}{die}{Propositionen}}%
                   {{die}{Proposition}{die}{Propositionen}}%
                   {{der}{Proposition}{den}{Propositionen}}%
                   {{der}{Proposition}{der}{Propositionen}},
    satz       = {nach}{{der}{Satz}{die}{Sätze}}%S\"atze is fine here too
                   {{den}{Satz}{die}{Sätze}}%    ditto
                   {{dem}{Satz}{den}{Sätzen}}%   S\"atzen is okay
                   {{des}{Satzes}{der}{Sätze}},% S\"atze is fine
    theorem    = {nach}{{das}{Theorem}{die}{Theoreme}}%
                   {{das}{Theorem}{die}{Theoreme}}%
                   {{dem}{Theorem}{den}{Theoremen}}%
                   {{des}{Theorems}{der}{Theoreme}},
    lemma      = {nach}{{das}{Lemma}{die}{Lemmata}}%
                   {{das}{Lemma}{die}{Lemmata}}%
                   {{dem}{Lemma}{den}{Lemmata}}%
                   {{des}{Lemmas}{der}{Lemmata}},
    definition = {nach}{{die}{Definition}{die}{Definitionen}}%
                   {{die}{Definition}{die}{Definitionen}}%
                   {{der}{Definition}{den}{Definitionen}}%
                   {{der}{Definition}{der}{Definitionen}},
  },
  % The keys (left side) must be written in LICR representation (see above).
  composition table for prepositions and articles = {
    an dem = am,
    in dem = im,
    von dem = vom,
    zu dem = zum,
    zu der = zur,
    bei dem = beim,
  },
  % The keys (left side) must be written in LICR representation (see above).
  prepositions always followed by the same case = {
    bis   = Akkusativ,
    durch = Akkusativ,
    f\"ur = Akkusativ,
    gegen = Akkusativ,
    ohne  = Akkusativ,
    um    = Akkusativ,
    aus  = Dativ,
    bei  = Dativ,
    mit  = Dativ,
    nach = Dativ,
    seit = Dativ,
    von  = Dativ,
    zu   = Dativ,
  },
}
\fi

% Rather than setting all of '/xcref/ngerman/names table' in one go, it is
% possible to define language-specific data for the various reference types
% one by one. If data for the specified reference type already exists, it is
% overwritten.
\xcrefngermanSetNamesTableEntry{section}{in}
  {{der}{Abschnitt}{die}{Abschnitte}}
  {{den}{Abschnitt}{die}{Abschnitte}}
  {{dem}{Abschnitt}{den}{Abschnitten}}
  {{des}{Abschnitts}{der}{Abschnitte}} % “des Abschnittes” is also possible

\begin{document}

\section{Dummy text with labels}
\label{sec-dummy-text-with-labels}

\begin{problem}\label{prob-a}
Here, we have \verb|\label{prob-a}|.
\end{problem}
\begin{problem}\label{prob-b}
Here, we have \verb|\label{prob-b}|.
\end{problem}
\begin{problem}\label{prob-c}
Here, we have \verb|\label{prob-c}|.
\end{problem}
\begin{problem}\label{prob-d}
Here, we have \verb|\label{prob-d}|.
\end{problem}
\begin{satz}\label{satz-a}
Here, we have \verb|\label{satz-a}|.
\end{satz}
\begin{satz}\label{satz-b}
Here, we have \verb|\label{satz-b}|.
\end{satz}
\begin{satz}\label{satz-c}
Here, we have \verb|\label{satz-c}|.
\end{satz}
\begin{proposition}\label{propo}
Here, we have \verb|\label{propo}|.
\end{proposition}

\section{Important remarks}
\label{sec-important-remarks}

Let's use accusative for all of the following, except (this is automatic) when
a preposition is \emph{always} used in combination with a particular case
(this happens in particular for “aus”, “bei”, “mit”, “nach”, “seit”, “von” and
“zu”, which are always followed by Dative). This is done with
\verb|\xcrefset{ngerman/case=acc}|. If no such command is performed, one has
to pass \verb|\xcref| the \verb|ngerman/case| option every time for German, as
in \verb|\xcref[ngerman/case=dat]{...}| or, using a German-specific helper
command, \verb|\xcrefngermanwrapper[case=dat]{...}| (valid values for
\texttt{/xcref/ngerman/case} are \verb|nom|, \verb|acc|, \verb|dat|, and
\verb|gen|).

\emph{I know that some of the following phrases sound weird or wrong; these
  are just artificial examples to illustrate how \textsf{xcref} can be used.}

Just to show the effect of our call to \verb|\xcrefngermanSetNamesTableEntry|:
the items from section~\ref{sec-dummy-text-with-labels} will be referenced
with \textsf{xcref} \emph{\gxcref[case=dat]{sec-examples}} or
\emph{\gxcref[preposition=durch]{sec-examples}}. Once can also use the
\verb|prep+noun| form: \emph{%
  \gxcref[form=prep+noun, case=dat]{sec-dummy-text-with-labels},
  \gxcref[form=prep+noun, case=dat]{sec-dummy-text-with-labels,sec-examples},
  \gxcref[form=prep+noun, case=dat]{%
    sec-dummy-text-with-labels,sec-examples,sec-important-remarks}}.

\section{Examples}
\label{sec-examples}

Basic test with \verb|\cref| and \verb|\Cref|: \cref{satz-a}
(obtained with \verb|\cref{satz-a}|), \Cref{satz-a} (obtained with
\verb|\Cref{satz-a}|). Note that we have used:
\begin{verbatim}
\creflabelformat{satz}{#2(#1)#3}
\end{verbatim}
in order to show (see below) that \textsf{xcref} preserves this formatting
specification. This explains the parentheses around “Satz” numbers in
references.

% This command respects TeX's grouping rules
\xcrefset{ngerman/case=acc}% use accusative by default
%
\verb|xcref-ngerman.tex| knows that “nach” is always followed by Dative:
\gxcref[preposition=nach]{prob-a,prob-b}.\\
\xcref[hyperlinks=false]{prob-a,prob-b}~\texttt{==}~%
{% Other way to to the same:
 \xcrefset{hyperlinks=false}% lasts for the duration of the current group
 \gxcref{prob-a,prob-b}%
}\\
\gXcref{prob-a}\\
\gxcref[form=article+noun]{prob-a,prob-b}\\
\gxcref[form=noun]{prob-a,prob-b}

% Turn off hyperlinks from now on
\xcrefset{hyperlinks=false}
\gxcref[form=noun]{prob-a,prob-b,prob-c,prob-d,satz-a,satz-b}\\
\gxcref[form=article+noun]{prob-a,prob-b,prob-c,prob-d,satz-a,satz-b}\\
\gxcref{prob-a,prob-b,prob-c,prob-d,satz-a,satz-b}\\
% Override the default prepositions for this call
\gxcref[preposition=in]{prob-a,prob-b,prob-c,prob-d,satz-a,satz-b}\\
\gXcref{prob-a,prob-b,prob-c,prob-d,propo,satz-a,satz-b}\\
\gxcref[preposition=von]{satz-a}, \gxcref[preposition=aus]{satz-b},
\gxcref[preposition=durch]{satz-a}

\end{document}
